\clearpage
\part*{Introduction}

	Une intro

	\clearpage

	% Exemple de code pour ajouter une figure
	% \begin{figure}[h]
	% 	\centering
	% 	\includegraphics[scale=0.8]{imgs/IMAGE.jpg}
	% 	\caption{Définitions géographique de l'espace arctique}
	% 	\label{def_esp_arctic} % le label sert à citer cette figure dans le texte
	%														C'est une genre de référence
	% \end{figure}

	\newpage

\newpage
\part{Introduction à la cryptographie légère}


	\section{Généralités}



\part{Présentation de deux algorithmes de cryptographie légère}

	\section{SPONGENT}

		\subsection{Fonctionnement général de SPONGENT}

		\subsection{Fonctionnement détaillé de SPONGENT 160 / 160 / 80}


		\subparagraph{Performances et sécurité}

	\section{Speck}

			Speck est un algorithme de chiffrement par bloc léger crée par la NSA et rendu
		publique en 2013. C'est un algorithme spécialement conçu pour avoir des performances
		élevées afin d'offrir un algorithme de chiffrement utilisable dans le cadre de
		"l'Internet of Things".

		\subsection{Algorithmes ARX}

				Cet algorithme fait parti des algorithmes dits ARX, Add-Rotate-Xor. C'est une famille
			d'algorithmes qui n'utilisent que les opérations d'additions, rotations et ou exclusif
			dans l'espace $GF_{2^n}$. Il y a plusieurs avantages à se limiter à ces opérations:

			\begin{enumerate}
			\item[•] Rapidité: ces opérations sont des opérations logiques. Ainsi, elles sont
				des primitives de tout micro-controlleur et donc sont effectué en un seul
				cycle d'horloge.
			\item[•] Sécurité matérielle: le fait que toutes les opérations soient des opérations
				logiques atomiques permet à ces algorithmes de fonctionner en temps constant.
				Prévenant les attaques par canaux cachés basés sur les mesures de temps.
			\item[•] Implémentation: ces algorithmes sont souvent très simples. Leur implémentation
				qu'elle soit logicielle ou matérielle est très simple. Par conséquent, le
				temps de développement et le coût de leur implémentation est très faible.
			\end{enumerate}

		\subsection{Fonctionnement de Speck}

			Avant de détailler comment fonctionne Speck, considérons les notations suivantes:

			\begin{enumerate}
			  \item[•] Le ou-exclusif bit à bit, noté xor
			  \item[•] L'addiction modulo $2^n$, noté $\xor$
			  \item[•] Les rotations circulaires à gauche et à droite respectivement notées,
			    $S^i$ et $S^{-i}$ pour des rotations de i-bits.
			\end{enumerate}

			Un tour de chiffrement de l'algorithme Speck est définit de la façon suivante. \\
			Pour $k \in GF(2^n)$ une clée, le tour de chiffrement est définit par la fonction suivante:
			\[
			\begin{array}{ccccc}
			R_k & : & GF(2^n) x GF(2^n) & \to & GF(2^n) x GF(2^n) \\
			 & & (x,y) & \mapsto & ((S^{-\alpha}(x) + y) \xor k, S^\beta (y) \xor (S^{-\alpha} + y) \xor k) \\
			\end{array}
			\]

			avec $\alpha$ = 7 et $\beta$ = 2 si n = 16, $\alpha $= 8 et $\beta$ = 3 sinon. \\

			La fonction de déchiffrement est définie par:
			\[
			\begin{array}{ccccc}
			R_k^{-1} & : & GF(2^n) x GF(2^n) & \to & GF(2^n) x GF(2^n) \\
			 & & (x,y) & \mapsto & (S^\alpha ((x \xor k) - S^{-\beta}(x \xor y)), S^{-\beta}(x \xor y)) \\
			\end{array}
			\]


\newpage
\part*{Conclusion}

		La conlusion
